\documentclass{article}
\usepackage{listings}
\begin{document}
	\title{PS1 Answers}
	\author{Anna Voss}
	\maketitle
	\begin{lstlisting}
	##Question 1
	y <- c(105, 69, 86, 100, 82, 111, 104, 110, 87, 108, 87, 90, 94, 113, 112, 98, 80, 97, 95, 111, 114, 89, 95, 126, 98)
	#Use qnorm (n > 30)
	z90 <- qnorm((1- 0.90)/2, lower.tail = FALSE)
	n = length(y)
	sample_mean <- mean(y)
	sample_sd <- sd(y)
	lower_90 <- sample_mean - (z90 * (sample_sd/sqrt(n)))
	upper_90 <- sample_mean + (z90 * (sample_sd/sqrt(n))) 
	confint90 <- c(lower_90, upper_90)
	confint90
	# [94.13283,102.74717]
	
	##Question 2 
	y <- c(105, 69, 86, 100, 82, 111, 104, 110, 87, 108, 87, 90, 94, 113, 112, 98, 80, 97, 95, 111, 114, 89, 95, 126, 98) 
	##Data Normally distributed so can use 1 sample t-test 
	t.test(y, mu = 100)
	# 	One Sample t-test
	# t = -0.59574, df = 24, p-value = 0.5569
	# alternative hypothesis: true mean is not equal to 100
	# 95 percent confidence interval:
	#   93.03553 103.84447
	# sample estimates:
	#   mean of x 
	# 98.44 
	
	##Question 3 
	expenditure <- read.table("expenditure.txt", header=TRUE) 
	
	#a
	library("tidyverse")
	qplot(x = X1, y = Y, data = expenditure)
	#X1 and Y show a strong positive correlation. On average, as per capita expenditure on public education increases, per capita personal income increases
	qplot(x = X2, y = Y, data = expenditure)
	#X2 and Y have a linear correlation. The number of resident per thousand remains relatively constant and the per capita expenditure on public education increases.
	qplot(x = X3, y = Y, data = expenditure)
	#X3 and Y have a postive correlation. On average, the number of people per thousand residing in urban area increases as per capita expenditure on public education increases.
	
	#b
	ggplot(expenditure, aes(Region, Y)) + 
	geom_boxplot(aes(group=Region))
	#Region 4 (West) has the highest per capita expenditure on public education 
	
	#c 
	qplot(x = X1, y = Y, data = expenditure)
	#X1 and Y show a strong positive correlation. On average, as per capita expenditure on public education increases, per capita personal income increases
	qplot(x = X1, y = Y, data = expenditure, shape = Region)
	
	ggplot(expenditure, aes(X1, Y)) +
	geom_point(aes(shape = Region, color = Region)) + 
	scale_shape_identity() + 
	scale_color_gradient(low="blue", high="red")
	\end{lstlisting}
\end{document}